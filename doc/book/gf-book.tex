\documentstyle[isolatin1, 12pt]{report}
%\setlength{\parindent}{0mm}
%\setlength{\parskip}{2mm}
\setlength{\textheight}{240mm}
\setlength{\textwidth}{148mm}
\setlength{\topmargin}{-16mm}

\input{macros}

\renewcommand{\keyw}[1]{\mbox{{\bf #1 }}} % keywords in bold
\newcommand{\syb}[1]{{\sl #1}}            % identifiers in slanted romans
\newcommand{\str}[1]{{\em ``#1"}}         % strings in italics (with quotes)
\newcommand{\kwtable}{\mbox{{\bf table}}} % the word table (sometimes nice to omit)
\newcommand{\gfsize}{\small}              % gf code is written in small

\newcommand{\begGF}{\begin{verbatim}}  % for judgements
\newcommand{\begTGF}{\begin{verbatim}} % for terms

\renewcommand{\eop}[1]{\syb{#1}}

\newcommand{\keyword}[1]{\mbox{{\bf #1}}}
\newcommand{\cs}[1]{#1}
\newcommand{\tm}[1]{"\mbox{#1}"}
\newcommand{\maincomment}[1]{}
\newcommand{\lin}{\keyword{lin}}
\newcommand{\lincat}{\keyword{lincat}}
\newcommand{\oper}{\keyword{oper}}
\newcommand{\param}{\keyword{param}}
\newcommand{\tplus}{+\!\!+\,}
\newcommand{\sel}{\; ! \;}
\newcommand{\funarr}{\rightarrow}
\newcommand{\Str}{\eop{Str}}
\newcommand{\catcat}{\eop{cat}}
\newcommand{\fun}{\keyword{fun}}
\newcommand{\tbl}{\keyword{table}}
\newcommand{\pattern}{\keyword{pattern}}
\newcommand{\equal}{\keyword{ =}}
\renewcommand{\colon}{\keyword{ :}}
\newcommand{\plusplus}{\mbox{ \scriptsize{$+\!+$} }}
\newcommand{\pp}{\plusplus}

\newcommand{\linn}[2]{#1 \; \rhd \; #2}
\newcommand{\abstr}[2]{\lambda #1 \rightarrow #2}
\newcommand{\LiT}[1]{\sugmap{#1}}
\newcommand{\letexp}[4]{\keyword{let} \; \{ #1 = #2 : #3 \} \; \keyword{in} \; #4}
\newcommand{\metav}[2]{?_{\eop{\tiny #1}_{\scriptsize #2}}}

\newcommand{\brandnewslide}[1]{\section{#1}}
\newcommand{\newslide}[1]{\subsection{#1}}
\newcommand{\oldslide}[1]{}

\newcommand{\empha}[1]{{\em #1}}


\title{{\Large \bf Building Natural-Language Applications in GF}}

\author{Aarne Ranta}

\begin{document}

%%%%%%%%%%%%%%%%%%%%%%%%%%%%%%%%%%%%%%%%%%%%%%%%%%%%%%%%%%%%%%%%%%%%

\maketitle



\chapter*{Introduction}

This book is written for programmers who want to build 
natural-language applications without first having to acquire
professional training in linguistics. The aim with making this
happen is to make natural-language applications more wide-spread
and to improve their quality. The tool that makes this possible
is the Grammatical Framework, GF.

There are many reasons why natural-language applications are not
common. One reason is that they are difficult to build. No-one has
ever managed to write a program that correctly recognizes arbitrary 
natural-language input in any language.


GF is a special-purpose programming language for writing grammars.
Programs written in GF are called grammars. A grammar is a declarative
description of a language, which can be a natural language (like
English or Finnish) or a formal language (like C or predicate logic).





\chapter{Building a Phrase Book}

\chapter{Using a Linguistic Resource}





\bibliographystyle{plain}

\bibliography{gf-bib}


\end{document}
